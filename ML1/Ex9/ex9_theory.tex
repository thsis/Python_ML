\documentclass[12pt]{article}
\usepackage{amsmath}
\usepackage{amssymb}
\usepackage{bbm}
\usepackage{mathtools}
\usepackage{fancyhdr}
\pagestyle{fancy}
\setlength{\headheight}{15pt}
\lhead{\bfseries Group: SHPZKN}
\rhead{\bfseries ML Homework 9}

\newcommand{\mysum}[2]{\sum_{#1 = 1}^{n}  #2}
\newcommand{\kersum}[1]{\sum_{#1 = 1}^{n}  \alpha_{#1} y_{#1} x_{#1}}
\newcommand{\dksum}[2]{\sum_{#1=1}^{n}  \sum_{#2=1}^{n}  
                                              \alpha_{#1}\alpha_{#2}
                                              y_{#1} y_{#2} 
                                              x_{#1}^{T} x_{#2}}

\newcommand{\matcell}[2]{y_{#1} y_{#2} x_{#1}^T x_{#2}}
\begin{document}
\section*{Problem 1}

\begin{equation}
min {\lVert w\rVert}^2 \; s.t.: \;  y_i\left(w^Tx_i + \theta\right)  \geq 1, \forall i \in \{1,\; ...,\; n\}
\end{equation}

\subsection*{a)}
Formulate the Lagrangian:
\begin{equation}
\label{primal}
\Lambda\left(w, \theta, \alpha\right)=\frac{1}{2}{\lVert w\rVert}^2 + 
    \mysum{i}{\alpha_i \left[y_i\left(w^Tx_i + \theta\right) - 1\right]}
\end{equation}

\subsection*{b)}
Partial Derivatives

\begin{equation}
\label{prt1}
\frac{\Lambda\left(w, \theta, \alpha\right)}{\partial w} 
    = w + \mysum{i}{\alpha_i y_i x_i} \stackrel{!}{=} 0 
    \Rightarrow w = - \kersum{i}
\end{equation}

\begin{equation}
\label{prt2}
\frac{\Lambda\left(w, \theta, \alpha\right)}{\partial \theta} 
    = \mysum{i}{\alpha_i y_i} \stackrel{!}{=} 0 
\end{equation}

Substitute the results from the partial derivatives \eqref{prt1} and \eqref{prt2} into the primal
\eqref{primal} to get the dual problem:
\begin{equation}
\label{dual_ini}
\begin{split}
\Lambda\left(w, \theta, \alpha\right) 
& =\frac{1}{2}{\lVert w\rVert}^2 + 
        \mysum{i}{\alpha_i \left[y_i\left(w^Tx_i + \theta\right) - 1\right]} \\
&= \frac{1}{2} {\lVert \kersum{i} \rVert}^2 
       - \mysum{k}{\alpha_k \left[y_k\left({\left(\kersum{l}\right)}^T x_k + \theta\right)-1 \right]} \\
&= \frac{1}{2} \dksum{i}{j} - \dksum{k}{l} + \sum_{k}^{n} \alpha_k y_k \theta 
      + \sum_{k}^{n} \alpha_k \\
&= -\frac{1}{2} \dksum{i}{j} + \theta \underbrace{\sum_k^n \alpha_k y_k}_{=0} 
      + \sum_{k}^{n} \alpha_k \\
&= - \frac{1}{2} \dksum{i}{j} + \sum_k^n \alpha_k \stackrel{!}{=} L\left(\alpha\right)
\end{split}
\end{equation}
The dual problem can therefore be expressed as:
\begin{equation}
\label{dual_proper}
\begin{split}
max  - \frac{1}{2} \dksum{i}{j} + \sum_k^n \alpha_k \\
s.t. \;  \kersum{i} \stackrel{!}{=} 0\; and \; \alpha_i \geq 0 \; \forall i \in \{1,\; ...,\; n\}
\end{split}
\end{equation}
Which is now a standard optimization problem in $\alpha$. 


\subsection*{c)}
Primal - kernelized:
\begin{equation}
\label{pri_ker}
\begin{split}
& min\;\frac{1}{2}{\lVert w\rVert}^2 + 
    \mysum{i}{\alpha_i \left[y_i\left(w^Tx_i + \theta\right) - 1\right]} \\
&= min \; k\left(w^Tw\right) + 
    \mysum{i}{\alpha_i \left[y_i\left(w^Tx_i + \theta\right) - 1\right]}
\end{split}
\end{equation}
Dual - kernelized:
\begin{equation}
\begin{split}
& max\; - \frac{1}{2} \dksum{i}{j} + \sum_k^n \alpha_k \\
& max\; - \frac{1}{2} k\left(w^Tw\right) + \sum_k^n \alpha_k
\end{split}
\end{equation}

\section*{Problem 2}
Reformulate the dual problem \eqref{dual_proper}
\begin{equation}
\begin{split}
min\; \frac{1}{2} \dksum{i}{j} - \sum_k^n \alpha_k \\
s.t. \;  \kersum{i} \stackrel{!}{=} 0\; and \; \alpha_i \geq 0 \; \forall i \in \{1,\; ...,\; n\}
\end{split}
\end{equation}
\newpage
According to this we can formulate 
\begin{enumerate}
\item{the quadratic form as:}
\begin{equation}
\begin{split}
x^T P x 
& \Rightarrow \dksum{i}{j} \\ 
&= \alpha^T  
    \begin{bmatrix}
        \matcell{1}{1} & \dots       & \matcell{1}{n} \\
        \vdots               & \ddots     & \vdots \\
        \matcell{n}{1} & \dots       & \matcell{n}{n}
\end{bmatrix} \alpha \\
&= \alpha^T \left[yy^T * XX^T X\right] \alpha \\
\end{split}
\end{equation}
where $*$ denotes element-wise multiplication.
\item{the scalar-product as:}
\begin{equation}
q^T x \Rightarrow - \sum_k^n \alpha_k = - \mathbbm{1}^T \alpha
\end{equation}
 where $\mathbbm{1}$ denotes the vector of ones 
\item{the first constraint:}
\begin{equation}
Gx \preceq h \Rightarrow - \alpha \preceq 0
\end{equation}
\item{the second constraint:}
\begin{equation}
Ax = b \Rightarrow y^T \alpha = \vec{0}
\end{equation}
\end{enumerate}

To sum up:
\begin{enumerate}
\item{$P = \left[yy^T * XX^T X\right]$}
\item{$q = - \mathbbm{1}$}
\item{$G = -1$}
\item{$h = 0$}
\item{$A = y^T$}
\item{$b = \vec{0}$}
\end{enumerate}
\end{document}