
% Default to the notebook output style

    


% Inherit from the specified cell style.




    
\documentclass[11pt]{article}

    
    
    \usepackage[T1]{fontenc}
    % Nicer default font (+ math font) than Computer Modern for most use cases
    \usepackage{mathpazo}

    % Basic figure setup, for now with no caption control since it's done
    % automatically by Pandoc (which extracts ![](path) syntax from Markdown).
    \usepackage{graphicx}
    % We will generate all images so they have a width \maxwidth. This means
    % that they will get their normal width if they fit onto the page, but
    % are scaled down if they would overflow the margins.
    \makeatletter
    \def\maxwidth{\ifdim\Gin@nat@width>\linewidth\linewidth
    \else\Gin@nat@width\fi}
    \makeatother
    \let\Oldincludegraphics\includegraphics
    % Set max figure width to be 80% of text width, for now hardcoded.
    \renewcommand{\includegraphics}[1]{\Oldincludegraphics[width=.8\maxwidth]{#1}}
    % Ensure that by default, figures have no caption (until we provide a
    % proper Figure object with a Caption API and a way to capture that
    % in the conversion process - todo).
    \usepackage{caption}
    \DeclareCaptionLabelFormat{nolabel}{}
    \captionsetup{labelformat=nolabel}

    \usepackage{adjustbox} % Used to constrain images to a maximum size 
    \usepackage{xcolor} % Allow colors to be defined
    \usepackage{enumerate} % Needed for markdown enumerations to work
    \usepackage{geometry} % Used to adjust the document margins
    \usepackage{amsmath} % Equations
    \usepackage{amssymb} % Equations
    \usepackage{textcomp} % defines textquotesingle
    % Hack from http://tex.stackexchange.com/a/47451/13684:
    \AtBeginDocument{%
        \def\PYZsq{\textquotesingle}% Upright quotes in Pygmentized code
    }
    \usepackage{upquote} % Upright quotes for verbatim code
    \usepackage{eurosym} % defines \euro
    \usepackage[mathletters]{ucs} % Extended unicode (utf-8) support
    \usepackage[utf8x]{inputenc} % Allow utf-8 characters in the tex document
    \usepackage{fancyvrb} % verbatim replacement that allows latex
    \usepackage{grffile} % extends the file name processing of package graphics 
                         % to support a larger range 
    % The hyperref package gives us a pdf with properly built
    % internal navigation ('pdf bookmarks' for the table of contents,
    % internal cross-reference links, web links for URLs, etc.)
    \usepackage{hyperref}
    \usepackage{longtable} % longtable support required by pandoc >1.10
    \usepackage{booktabs}  % table support for pandoc > 1.12.2
    \usepackage[inline]{enumitem} % IRkernel/repr support (it uses the enumerate* environment)
    \usepackage[normalem]{ulem} % ulem is needed to support strikethroughs (\sout)
                                % normalem makes italics be italics, not underlines
    

    
    
    % Colors for the hyperref package
    \definecolor{urlcolor}{rgb}{0,.145,.698}
    \definecolor{linkcolor}{rgb}{.71,0.21,0.01}
    \definecolor{citecolor}{rgb}{.12,.54,.11}

    % ANSI colors
    \definecolor{ansi-black}{HTML}{3E424D}
    \definecolor{ansi-black-intense}{HTML}{282C36}
    \definecolor{ansi-red}{HTML}{E75C58}
    \definecolor{ansi-red-intense}{HTML}{B22B31}
    \definecolor{ansi-green}{HTML}{00A250}
    \definecolor{ansi-green-intense}{HTML}{007427}
    \definecolor{ansi-yellow}{HTML}{DDB62B}
    \definecolor{ansi-yellow-intense}{HTML}{B27D12}
    \definecolor{ansi-blue}{HTML}{208FFB}
    \definecolor{ansi-blue-intense}{HTML}{0065CA}
    \definecolor{ansi-magenta}{HTML}{D160C4}
    \definecolor{ansi-magenta-intense}{HTML}{A03196}
    \definecolor{ansi-cyan}{HTML}{60C6C8}
    \definecolor{ansi-cyan-intense}{HTML}{258F8F}
    \definecolor{ansi-white}{HTML}{C5C1B4}
    \definecolor{ansi-white-intense}{HTML}{A1A6B2}

    % commands and environments needed by pandoc snippets
    % extracted from the output of `pandoc -s`
    \providecommand{\tightlist}{%
      \setlength{\itemsep}{0pt}\setlength{\parskip}{0pt}}
    \DefineVerbatimEnvironment{Highlighting}{Verbatim}{commandchars=\\\{\}}
    % Add ',fontsize=\small' for more characters per line
    \newenvironment{Shaded}{}{}
    \newcommand{\KeywordTok}[1]{\textcolor[rgb]{0.00,0.44,0.13}{\textbf{{#1}}}}
    \newcommand{\DataTypeTok}[1]{\textcolor[rgb]{0.56,0.13,0.00}{{#1}}}
    \newcommand{\DecValTok}[1]{\textcolor[rgb]{0.25,0.63,0.44}{{#1}}}
    \newcommand{\BaseNTok}[1]{\textcolor[rgb]{0.25,0.63,0.44}{{#1}}}
    \newcommand{\FloatTok}[1]{\textcolor[rgb]{0.25,0.63,0.44}{{#1}}}
    \newcommand{\CharTok}[1]{\textcolor[rgb]{0.25,0.44,0.63}{{#1}}}
    \newcommand{\StringTok}[1]{\textcolor[rgb]{0.25,0.44,0.63}{{#1}}}
    \newcommand{\CommentTok}[1]{\textcolor[rgb]{0.38,0.63,0.69}{\textit{{#1}}}}
    \newcommand{\OtherTok}[1]{\textcolor[rgb]{0.00,0.44,0.13}{{#1}}}
    \newcommand{\AlertTok}[1]{\textcolor[rgb]{1.00,0.00,0.00}{\textbf{{#1}}}}
    \newcommand{\FunctionTok}[1]{\textcolor[rgb]{0.02,0.16,0.49}{{#1}}}
    \newcommand{\RegionMarkerTok}[1]{{#1}}
    \newcommand{\ErrorTok}[1]{\textcolor[rgb]{1.00,0.00,0.00}{\textbf{{#1}}}}
    \newcommand{\NormalTok}[1]{{#1}}
    
    % Additional commands for more recent versions of Pandoc
    \newcommand{\ConstantTok}[1]{\textcolor[rgb]{0.53,0.00,0.00}{{#1}}}
    \newcommand{\SpecialCharTok}[1]{\textcolor[rgb]{0.25,0.44,0.63}{{#1}}}
    \newcommand{\VerbatimStringTok}[1]{\textcolor[rgb]{0.25,0.44,0.63}{{#1}}}
    \newcommand{\SpecialStringTok}[1]{\textcolor[rgb]{0.73,0.40,0.53}{{#1}}}
    \newcommand{\ImportTok}[1]{{#1}}
    \newcommand{\DocumentationTok}[1]{\textcolor[rgb]{0.73,0.13,0.13}{\textit{{#1}}}}
    \newcommand{\AnnotationTok}[1]{\textcolor[rgb]{0.38,0.63,0.69}{\textbf{\textit{{#1}}}}}
    \newcommand{\CommentVarTok}[1]{\textcolor[rgb]{0.38,0.63,0.69}{\textbf{\textit{{#1}}}}}
    \newcommand{\VariableTok}[1]{\textcolor[rgb]{0.10,0.09,0.49}{{#1}}}
    \newcommand{\ControlFlowTok}[1]{\textcolor[rgb]{0.00,0.44,0.13}{\textbf{{#1}}}}
    \newcommand{\OperatorTok}[1]{\textcolor[rgb]{0.40,0.40,0.40}{{#1}}}
    \newcommand{\BuiltInTok}[1]{{#1}}
    \newcommand{\ExtensionTok}[1]{{#1}}
    \newcommand{\PreprocessorTok}[1]{\textcolor[rgb]{0.74,0.48,0.00}{{#1}}}
    \newcommand{\AttributeTok}[1]{\textcolor[rgb]{0.49,0.56,0.16}{{#1}}}
    \newcommand{\InformationTok}[1]{\textcolor[rgb]{0.38,0.63,0.69}{\textbf{\textit{{#1}}}}}
    \newcommand{\WarningTok}[1]{\textcolor[rgb]{0.38,0.63,0.69}{\textbf{\textit{{#1}}}}}
    
    
    % Define a nice break command that doesn't care if a line doesn't already
    % exist.
    \def\br{\hspace*{\fill} \\* }
    % Math Jax compatability definitions
    \def\gt{>}
    \def\lt{<}
    % Document parameters
    \title{sheet4}
    
    
    

    % Pygments definitions
    
\makeatletter
\def\PY@reset{\let\PY@it=\relax \let\PY@bf=\relax%
    \let\PY@ul=\relax \let\PY@tc=\relax%
    \let\PY@bc=\relax \let\PY@ff=\relax}
\def\PY@tok#1{\csname PY@tok@#1\endcsname}
\def\PY@toks#1+{\ifx\relax#1\empty\else%
    \PY@tok{#1}\expandafter\PY@toks\fi}
\def\PY@do#1{\PY@bc{\PY@tc{\PY@ul{%
    \PY@it{\PY@bf{\PY@ff{#1}}}}}}}
\def\PY#1#2{\PY@reset\PY@toks#1+\relax+\PY@do{#2}}

\expandafter\def\csname PY@tok@w\endcsname{\def\PY@tc##1{\textcolor[rgb]{0.73,0.73,0.73}{##1}}}
\expandafter\def\csname PY@tok@c\endcsname{\let\PY@it=\textit\def\PY@tc##1{\textcolor[rgb]{0.25,0.50,0.50}{##1}}}
\expandafter\def\csname PY@tok@cp\endcsname{\def\PY@tc##1{\textcolor[rgb]{0.74,0.48,0.00}{##1}}}
\expandafter\def\csname PY@tok@k\endcsname{\let\PY@bf=\textbf\def\PY@tc##1{\textcolor[rgb]{0.00,0.50,0.00}{##1}}}
\expandafter\def\csname PY@tok@kp\endcsname{\def\PY@tc##1{\textcolor[rgb]{0.00,0.50,0.00}{##1}}}
\expandafter\def\csname PY@tok@kt\endcsname{\def\PY@tc##1{\textcolor[rgb]{0.69,0.00,0.25}{##1}}}
\expandafter\def\csname PY@tok@o\endcsname{\def\PY@tc##1{\textcolor[rgb]{0.40,0.40,0.40}{##1}}}
\expandafter\def\csname PY@tok@ow\endcsname{\let\PY@bf=\textbf\def\PY@tc##1{\textcolor[rgb]{0.67,0.13,1.00}{##1}}}
\expandafter\def\csname PY@tok@nb\endcsname{\def\PY@tc##1{\textcolor[rgb]{0.00,0.50,0.00}{##1}}}
\expandafter\def\csname PY@tok@nf\endcsname{\def\PY@tc##1{\textcolor[rgb]{0.00,0.00,1.00}{##1}}}
\expandafter\def\csname PY@tok@nc\endcsname{\let\PY@bf=\textbf\def\PY@tc##1{\textcolor[rgb]{0.00,0.00,1.00}{##1}}}
\expandafter\def\csname PY@tok@nn\endcsname{\let\PY@bf=\textbf\def\PY@tc##1{\textcolor[rgb]{0.00,0.00,1.00}{##1}}}
\expandafter\def\csname PY@tok@ne\endcsname{\let\PY@bf=\textbf\def\PY@tc##1{\textcolor[rgb]{0.82,0.25,0.23}{##1}}}
\expandafter\def\csname PY@tok@nv\endcsname{\def\PY@tc##1{\textcolor[rgb]{0.10,0.09,0.49}{##1}}}
\expandafter\def\csname PY@tok@no\endcsname{\def\PY@tc##1{\textcolor[rgb]{0.53,0.00,0.00}{##1}}}
\expandafter\def\csname PY@tok@nl\endcsname{\def\PY@tc##1{\textcolor[rgb]{0.63,0.63,0.00}{##1}}}
\expandafter\def\csname PY@tok@ni\endcsname{\let\PY@bf=\textbf\def\PY@tc##1{\textcolor[rgb]{0.60,0.60,0.60}{##1}}}
\expandafter\def\csname PY@tok@na\endcsname{\def\PY@tc##1{\textcolor[rgb]{0.49,0.56,0.16}{##1}}}
\expandafter\def\csname PY@tok@nt\endcsname{\let\PY@bf=\textbf\def\PY@tc##1{\textcolor[rgb]{0.00,0.50,0.00}{##1}}}
\expandafter\def\csname PY@tok@nd\endcsname{\def\PY@tc##1{\textcolor[rgb]{0.67,0.13,1.00}{##1}}}
\expandafter\def\csname PY@tok@s\endcsname{\def\PY@tc##1{\textcolor[rgb]{0.73,0.13,0.13}{##1}}}
\expandafter\def\csname PY@tok@sd\endcsname{\let\PY@it=\textit\def\PY@tc##1{\textcolor[rgb]{0.73,0.13,0.13}{##1}}}
\expandafter\def\csname PY@tok@si\endcsname{\let\PY@bf=\textbf\def\PY@tc##1{\textcolor[rgb]{0.73,0.40,0.53}{##1}}}
\expandafter\def\csname PY@tok@se\endcsname{\let\PY@bf=\textbf\def\PY@tc##1{\textcolor[rgb]{0.73,0.40,0.13}{##1}}}
\expandafter\def\csname PY@tok@sr\endcsname{\def\PY@tc##1{\textcolor[rgb]{0.73,0.40,0.53}{##1}}}
\expandafter\def\csname PY@tok@ss\endcsname{\def\PY@tc##1{\textcolor[rgb]{0.10,0.09,0.49}{##1}}}
\expandafter\def\csname PY@tok@sx\endcsname{\def\PY@tc##1{\textcolor[rgb]{0.00,0.50,0.00}{##1}}}
\expandafter\def\csname PY@tok@m\endcsname{\def\PY@tc##1{\textcolor[rgb]{0.40,0.40,0.40}{##1}}}
\expandafter\def\csname PY@tok@gh\endcsname{\let\PY@bf=\textbf\def\PY@tc##1{\textcolor[rgb]{0.00,0.00,0.50}{##1}}}
\expandafter\def\csname PY@tok@gu\endcsname{\let\PY@bf=\textbf\def\PY@tc##1{\textcolor[rgb]{0.50,0.00,0.50}{##1}}}
\expandafter\def\csname PY@tok@gd\endcsname{\def\PY@tc##1{\textcolor[rgb]{0.63,0.00,0.00}{##1}}}
\expandafter\def\csname PY@tok@gi\endcsname{\def\PY@tc##1{\textcolor[rgb]{0.00,0.63,0.00}{##1}}}
\expandafter\def\csname PY@tok@gr\endcsname{\def\PY@tc##1{\textcolor[rgb]{1.00,0.00,0.00}{##1}}}
\expandafter\def\csname PY@tok@ge\endcsname{\let\PY@it=\textit}
\expandafter\def\csname PY@tok@gs\endcsname{\let\PY@bf=\textbf}
\expandafter\def\csname PY@tok@gp\endcsname{\let\PY@bf=\textbf\def\PY@tc##1{\textcolor[rgb]{0.00,0.00,0.50}{##1}}}
\expandafter\def\csname PY@tok@go\endcsname{\def\PY@tc##1{\textcolor[rgb]{0.53,0.53,0.53}{##1}}}
\expandafter\def\csname PY@tok@gt\endcsname{\def\PY@tc##1{\textcolor[rgb]{0.00,0.27,0.87}{##1}}}
\expandafter\def\csname PY@tok@err\endcsname{\def\PY@bc##1{\setlength{\fboxsep}{0pt}\fcolorbox[rgb]{1.00,0.00,0.00}{1,1,1}{\strut ##1}}}
\expandafter\def\csname PY@tok@kc\endcsname{\let\PY@bf=\textbf\def\PY@tc##1{\textcolor[rgb]{0.00,0.50,0.00}{##1}}}
\expandafter\def\csname PY@tok@kd\endcsname{\let\PY@bf=\textbf\def\PY@tc##1{\textcolor[rgb]{0.00,0.50,0.00}{##1}}}
\expandafter\def\csname PY@tok@kn\endcsname{\let\PY@bf=\textbf\def\PY@tc##1{\textcolor[rgb]{0.00,0.50,0.00}{##1}}}
\expandafter\def\csname PY@tok@kr\endcsname{\let\PY@bf=\textbf\def\PY@tc##1{\textcolor[rgb]{0.00,0.50,0.00}{##1}}}
\expandafter\def\csname PY@tok@bp\endcsname{\def\PY@tc##1{\textcolor[rgb]{0.00,0.50,0.00}{##1}}}
\expandafter\def\csname PY@tok@fm\endcsname{\def\PY@tc##1{\textcolor[rgb]{0.00,0.00,1.00}{##1}}}
\expandafter\def\csname PY@tok@vc\endcsname{\def\PY@tc##1{\textcolor[rgb]{0.10,0.09,0.49}{##1}}}
\expandafter\def\csname PY@tok@vg\endcsname{\def\PY@tc##1{\textcolor[rgb]{0.10,0.09,0.49}{##1}}}
\expandafter\def\csname PY@tok@vi\endcsname{\def\PY@tc##1{\textcolor[rgb]{0.10,0.09,0.49}{##1}}}
\expandafter\def\csname PY@tok@vm\endcsname{\def\PY@tc##1{\textcolor[rgb]{0.10,0.09,0.49}{##1}}}
\expandafter\def\csname PY@tok@sa\endcsname{\def\PY@tc##1{\textcolor[rgb]{0.73,0.13,0.13}{##1}}}
\expandafter\def\csname PY@tok@sb\endcsname{\def\PY@tc##1{\textcolor[rgb]{0.73,0.13,0.13}{##1}}}
\expandafter\def\csname PY@tok@sc\endcsname{\def\PY@tc##1{\textcolor[rgb]{0.73,0.13,0.13}{##1}}}
\expandafter\def\csname PY@tok@dl\endcsname{\def\PY@tc##1{\textcolor[rgb]{0.73,0.13,0.13}{##1}}}
\expandafter\def\csname PY@tok@s2\endcsname{\def\PY@tc##1{\textcolor[rgb]{0.73,0.13,0.13}{##1}}}
\expandafter\def\csname PY@tok@sh\endcsname{\def\PY@tc##1{\textcolor[rgb]{0.73,0.13,0.13}{##1}}}
\expandafter\def\csname PY@tok@s1\endcsname{\def\PY@tc##1{\textcolor[rgb]{0.73,0.13,0.13}{##1}}}
\expandafter\def\csname PY@tok@mb\endcsname{\def\PY@tc##1{\textcolor[rgb]{0.40,0.40,0.40}{##1}}}
\expandafter\def\csname PY@tok@mf\endcsname{\def\PY@tc##1{\textcolor[rgb]{0.40,0.40,0.40}{##1}}}
\expandafter\def\csname PY@tok@mh\endcsname{\def\PY@tc##1{\textcolor[rgb]{0.40,0.40,0.40}{##1}}}
\expandafter\def\csname PY@tok@mi\endcsname{\def\PY@tc##1{\textcolor[rgb]{0.40,0.40,0.40}{##1}}}
\expandafter\def\csname PY@tok@il\endcsname{\def\PY@tc##1{\textcolor[rgb]{0.40,0.40,0.40}{##1}}}
\expandafter\def\csname PY@tok@mo\endcsname{\def\PY@tc##1{\textcolor[rgb]{0.40,0.40,0.40}{##1}}}
\expandafter\def\csname PY@tok@ch\endcsname{\let\PY@it=\textit\def\PY@tc##1{\textcolor[rgb]{0.25,0.50,0.50}{##1}}}
\expandafter\def\csname PY@tok@cm\endcsname{\let\PY@it=\textit\def\PY@tc##1{\textcolor[rgb]{0.25,0.50,0.50}{##1}}}
\expandafter\def\csname PY@tok@cpf\endcsname{\let\PY@it=\textit\def\PY@tc##1{\textcolor[rgb]{0.25,0.50,0.50}{##1}}}
\expandafter\def\csname PY@tok@c1\endcsname{\let\PY@it=\textit\def\PY@tc##1{\textcolor[rgb]{0.25,0.50,0.50}{##1}}}
\expandafter\def\csname PY@tok@cs\endcsname{\let\PY@it=\textit\def\PY@tc##1{\textcolor[rgb]{0.25,0.50,0.50}{##1}}}

\def\PYZbs{\char`\\}
\def\PYZus{\char`\_}
\def\PYZob{\char`\{}
\def\PYZcb{\char`\}}
\def\PYZca{\char`\^}
\def\PYZam{\char`\&}
\def\PYZlt{\char`\<}
\def\PYZgt{\char`\>}
\def\PYZsh{\char`\#}
\def\PYZpc{\char`\%}
\def\PYZdl{\char`\$}
\def\PYZhy{\char`\-}
\def\PYZsq{\char`\'}
\def\PYZdq{\char`\"}
\def\PYZti{\char`\~}
% for compatibility with earlier versions
\def\PYZat{@}
\def\PYZlb{[}
\def\PYZrb{]}
\makeatother


    % Exact colors from NB
    \definecolor{incolor}{rgb}{0.0, 0.0, 0.5}
    \definecolor{outcolor}{rgb}{0.545, 0.0, 0.0}



    
    % Prevent overflowing lines due to hard-to-break entities
    \sloppy 
    % Setup hyperref package
    \hypersetup{
      breaklinks=true,  % so long urls are correctly broken across lines
      colorlinks=true,
      urlcolor=urlcolor,
      linkcolor=linkcolor,
      citecolor=citecolor,
      }
    % Slightly bigger margins than the latex defaults
    
    \geometry{verbose,tmargin=1in,bmargin=1in,lmargin=1in,rmargin=1in}
    
    

    \begin{document}
    
    
    \maketitle
    
    

    
    \section{Sheet 4: Rounding, Overflow, Linear
Algebra}\label{sheet-4-rounding-overflow-linear-algebra}

In this exercise sheet, we look at various sources of numerical overflow
when executing Python and numpy code for large input values, and how to
efficiently handle them, for example, by using numpy special functions.

    \begin{Verbatim}[commandchars=\\\{\}]
{\color{incolor}In [{\color{incolor}1}]:} \PY{k+kn}{import} \PY{n+nn}{numpy}\PY{o}{,}\PY{n+nn}{utils}
\end{Verbatim}


    \subsection{Building a robust "softplus" nonlinear function (40
P)}\label{building-a-robust-softplus-nonlinear-function-40-p}

The softplus function is defined as:

\[
\mathrm{softplus}(x) = \log(1+\exp(x)).
\]

It intervenes as elementary computation in certain machine learning
models such as neural networks. Plotting it gives the following curve

where the function tends to zero for very negative input values and
tends to the identity for very positive input values.

    \begin{Verbatim}[commandchars=\\\{\}]
{\color{incolor}In [{\color{incolor}3}]:} \PY{k}{def} \PY{n+nf}{softplus}\PY{p}{(}\PY{n}{z}\PY{p}{)}\PY{p}{:} \PY{k}{return} \PY{n}{numpy}\PY{o}{.}\PY{n}{log}\PY{p}{(}\PY{l+m+mi}{1}\PY{o}{+}\PY{n}{numpy}\PY{o}{.}\PY{n}{exp}\PY{p}{(}\PY{n}{z}\PY{p}{)}\PY{p}{)}
\end{Verbatim}


    We consider an input vector from the module \texttt{utils} containing
varying values between 1 and 10000. We would like to apply the
\texttt{softplus} function to all of its element in an element-wise
manner.

    \begin{Verbatim}[commandchars=\\\{\}]
{\color{incolor}In [{\color{incolor}6}]:} \PY{n}{X} \PY{o}{=} \PY{n}{utils}\PY{o}{.}\PY{n}{softplus\PYZus{}inputs}
        \PY{n+nb}{print}\PY{p}{(}\PY{n}{X}\PY{p}{)}
\end{Verbatim}


    \begin{Verbatim}[commandchars=\\\{\}]
[-10000, -1000, -100, -10, -1, 0, 1, 10, 100, 1000, 10000]

    \end{Verbatim}

    We choose these large values in order to test whether the behavior of
the function is correct in all regimes of the function, in particular,
for very small or very large values. The code below applies the
\texttt{softplus} function directly to the vector of inputs and then
prints for all cases the input and the corresponding function output:

    \begin{Verbatim}[commandchars=\\\{\}]
{\color{incolor}In [{\color{incolor}7}]:} \PY{n}{Y} \PY{o}{=} \PY{n}{softplus}\PY{p}{(}\PY{n}{X}\PY{p}{)}
        \PY{k}{for} \PY{n}{x}\PY{p}{,}\PY{n}{y} \PY{o+ow}{in} \PY{n+nb}{zip}\PY{p}{(}\PY{n}{X}\PY{p}{,}\PY{n}{Y}\PY{p}{)}\PY{p}{:}
            \PY{n+nb}{print}\PY{p}{(}\PY{l+s+s1}{\PYZsq{}}\PY{l+s+s1}{softplus(}\PY{l+s+si}{\PYZpc{}11.4f}\PY{l+s+s1}{) = }\PY{l+s+si}{\PYZpc{}11.4f}\PY{l+s+s1}{\PYZsq{}}\PY{o}{\PYZpc{}}\PY{p}{(}\PY{n}{x}\PY{p}{,}\PY{n}{y}\PY{p}{)}\PY{p}{)}
\end{Verbatim}


    \begin{Verbatim}[commandchars=\\\{\}]
softplus(-10000.0000) =      0.0000
softplus( -1000.0000) =      0.0000
softplus(  -100.0000) =      0.0000
softplus(   -10.0000) =      0.0000
softplus(    -1.0000) =      0.3133
softplus(     0.0000) =      0.6931
softplus(     1.0000) =      1.3133
softplus(    10.0000) =     10.0000
softplus(   100.0000) =    100.0000
softplus(  1000.0000) =         inf
softplus( 10000.0000) =         inf

    \end{Verbatim}

    \begin{Verbatim}[commandchars=\\\{\}]
/usr/local/lib/python3.6/dist-packages/ipykernel\_launcher.py:1: RuntimeWarning: overflow encountered in exp
  """Entry point for launching an IPython kernel.

    \end{Verbatim}

    For large input values, the softplus function returns \texttt{inf}
whereas analysis of that function tells us that it should compute the
identity. Let's now try to apply the softplus function one element at a
time, to see whether the problem comes from numpy arrays:

    \begin{Verbatim}[commandchars=\\\{\}]
{\color{incolor}In [{\color{incolor}8}]:} \PY{k}{for} \PY{n}{x} \PY{o+ow}{in} \PY{n}{X}\PY{p}{:}
            \PY{n}{y} \PY{o}{=} \PY{n}{softplus}\PY{p}{(}\PY{n}{x}\PY{p}{)}
            \PY{n+nb}{print}\PY{p}{(}\PY{l+s+s1}{\PYZsq{}}\PY{l+s+s1}{softplus(}\PY{l+s+si}{\PYZpc{}11.4f}\PY{l+s+s1}{) = }\PY{l+s+si}{\PYZpc{}11.4f}\PY{l+s+s1}{\PYZsq{}}\PY{o}{\PYZpc{}}\PY{p}{(}\PY{n}{x}\PY{p}{,}\PY{n}{y}\PY{p}{)}\PY{p}{)}
\end{Verbatim}


    \begin{Verbatim}[commandchars=\\\{\}]
softplus(-10000.0000) =      0.0000
softplus( -1000.0000) =      0.0000
softplus(  -100.0000) =      0.0000
softplus(   -10.0000) =      0.0000
softplus(    -1.0000) =      0.3133
softplus(     0.0000) =      0.6931
softplus(     1.0000) =      1.3133
softplus(    10.0000) =     10.0000
softplus(   100.0000) =    100.0000
softplus(  1000.0000) =         inf
softplus( 10000.0000) =         inf

    \end{Verbatim}

    \begin{Verbatim}[commandchars=\\\{\}]
/usr/local/lib/python3.6/dist-packages/ipykernel\_launcher.py:1: RuntimeWarning: overflow encountered in exp
  """Entry point for launching an IPython kernel.

    \end{Verbatim}

    Unfortunately, the result is the same. We observe that the function
always stops working when its output approaches 1000, even though the
input was given in high precision \texttt{float64}.

    \begin{itemize}
\tightlist
\item
  Create an alternative function for \texttt{softplus} that applies to
  input scalars and that correctly applies to values that can be much
  larger than 1000 (e.g. billions or more). Your function can be written
  in Python directly and does not need numpy parallelization.
\end{itemize}

    \begin{Verbatim}[commandchars=\\\{\}]
{\color{incolor}In [{\color{incolor}9}]:} \PY{k}{def} \PY{n+nf}{softplus\PYZus{}stable}\PY{p}{(}\PY{n}{x}\PY{p}{)}\PY{p}{:}
            \PY{k}{if} \PY{n}{x} \PY{o}{\PYZgt{}}\PY{o}{=} \PY{l+m+mf}{10.0}\PY{p}{:}
                \PY{k}{return} \PY{n}{x}
            \PY{k}{else}\PY{p}{:}
                \PY{k}{return} \PY{n}{softplus}\PY{p}{(}\PY{n}{x}\PY{p}{)}
        
        \PY{k}{for} \PY{n}{x} \PY{o+ow}{in} \PY{n}{X}\PY{p}{:}
            \PY{n}{y} \PY{o}{=} \PY{n}{softplus\PYZus{}stable}\PY{p}{(}\PY{n}{x}\PY{p}{)}
            \PY{n+nb}{print}\PY{p}{(}\PY{l+s+s1}{\PYZsq{}}\PY{l+s+s1}{softplus\PYZus{}stable(}\PY{l+s+si}{\PYZpc{}11.4f}\PY{l+s+s1}{) = }\PY{l+s+si}{\PYZpc{}11.4f}\PY{l+s+s1}{\PYZsq{}}\PY{o}{\PYZpc{}}\PY{p}{(}\PY{n}{x}\PY{p}{,}\PY{n}{y}\PY{p}{)}\PY{p}{)}
\end{Verbatim}


    \begin{Verbatim}[commandchars=\\\{\}]
softplus\_stable(-10000.0000) =      0.0000
softplus\_stable( -1000.0000) =      0.0000
softplus\_stable(  -100.0000) =      0.0000
softplus\_stable(   -10.0000) =      0.0000
softplus\_stable(    -1.0000) =      0.3133
softplus\_stable(     0.0000) =      0.6931
softplus\_stable(     1.0000) =      1.3133
softplus\_stable(    10.0000) =     10.0000
softplus\_stable(   100.0000) =    100.0000
softplus\_stable(  1000.0000) =   1000.0000
softplus\_stable( 10000.0000) =  10000.0000

    \end{Verbatim}

    As we have seen in the previous exercise sheet, the problem of functions
that apply to scalars only is that they are less efficient than
functions that apply to vectors directly. Therefore, we would like to
handle the rounding issue directly at the vector level.

\begin{itemize}
\tightlist
\item
  Create a new softplus function that applies to vectors and that has
  the desired behavior for large input values. Your function should be
  fast for large input vectors (i.e. it is not appropriate to use an
  inner Python loop inside the function).
\end{itemize}

    \begin{Verbatim}[commandchars=\\\{\}]
{\color{incolor}In [{\color{incolor}71}]:} \PY{k}{def} \PY{n+nf}{softplus\PYZus{}stable\PYZus{}vec}\PY{p}{(}\PY{n}{x}\PY{p}{)}\PY{p}{:}
             \PY{c+c1}{\PYZsh{} Convert x to numpy array of float type, otherwise it may default to int. }
             \PY{n}{x} \PY{o}{=} \PY{n}{numpy}\PY{o}{.}\PY{n}{array}\PY{p}{(}\PY{n}{x}\PY{p}{,} \PY{n}{dtype}\PY{o}{=}\PY{n}{numpy}\PY{o}{.}\PY{n}{float}\PY{p}{)}
             \PY{n}{small\PYZus{}idx} \PY{o}{=} \PY{n}{numpy}\PY{o}{.}\PY{n}{less}\PY{p}{(}\PY{n}{x}\PY{p}{,} \PY{l+m+mi}{100}\PY{p}{)}
             \PY{c+c1}{\PYZsh{} Only update values which are smaller than 100.}
             \PY{n}{x}\PY{p}{[}\PY{n}{small\PYZus{}idx}\PY{p}{]} \PY{o}{=} \PY{n}{softplus}\PY{p}{(}\PY{n}{x}\PY{p}{[}\PY{n}{small\PYZus{}idx}\PY{p}{]}\PY{p}{)}
             
             \PY{k}{return} \PY{n}{x}
         
         \PY{n}{Y} \PY{o}{=} \PY{n}{softplus\PYZus{}stable\PYZus{}vec}\PY{p}{(}\PY{n}{X}\PY{p}{)}
         \PY{k}{for} \PY{n}{x}\PY{p}{,} \PY{n}{y} \PY{o+ow}{in} \PY{n+nb}{zip}\PY{p}{(}\PY{n}{X}\PY{p}{,} \PY{n}{Y}\PY{p}{)}\PY{p}{:}
             \PY{n+nb}{print}\PY{p}{(}\PY{l+s+s1}{\PYZsq{}}\PY{l+s+s1}{softplus\PYZus{}stable\PYZus{}vec(}\PY{l+s+si}{\PYZpc{}11.4f}\PY{l+s+s1}{) = }\PY{l+s+si}{\PYZpc{}11.4f}\PY{l+s+s1}{\PYZsq{}}\PY{o}{\PYZpc{}}\PY{p}{(}\PY{n}{x}\PY{p}{,}\PY{n}{y}\PY{p}{)}\PY{p}{)}
\end{Verbatim}


    \begin{Verbatim}[commandchars=\\\{\}]
softplus\_stable\_vec(-10000.0000) =      0.0000
softplus\_stable\_vec( -1000.0000) =      0.0000
softplus\_stable\_vec(  -100.0000) =      0.0000
softplus\_stable\_vec(   -10.0000) =      0.0000
softplus\_stable\_vec(    -1.0000) =      0.3133
softplus\_stable\_vec(     0.0000) =      0.6931
softplus\_stable\_vec(     1.0000) =      1.3133
softplus\_stable\_vec(    10.0000) =     10.0000
softplus\_stable\_vec(   100.0000) =    100.0000
softplus\_stable\_vec(  1000.0000) =   1000.0000
softplus\_stable\_vec( 10000.0000) =  10000.0000

    \end{Verbatim}

    \subsection{Computing a partition function (30
P)}\label{computing-a-partition-function-30-p}

We consider a discrete probability distribution of type \[
p(\boldsymbol{x};\boldsymbol{w}) = \frac{1}{Z(\boldsymbol{w})} \exp(\boldsymbol{x}^\top \boldsymbol{w})
\] where \(\boldsymbol{x} \in \{-1,1\}^{10}\) is an observation, and
\(\boldsymbol{w} \in \mathbb{R}^{10}\) is a vector of parameters. The
term \(Z(\boldsymbol{w})\) is called the partition function and is
chosen such that the probability distribution sums to 1. That is, the
equation: \[
\sum_{\boldsymbol{x} \in \{-1,1\}^{10}} p(\boldsymbol{x};\boldsymbol{w}) = 1
\] must be satisfied. Below is a simple method that computes the log of
the partition function \(Z(\boldsymbol{w})\) for various choices of
parameter vectors. The considered parameters (\texttt{w\_small},
\texttt{w\_medium}, and \texttt{w\_large}) are increasingly large (and
thus problematic), and can be found in the file \texttt{utils.py}.

    \begin{Verbatim}[commandchars=\\\{\}]
{\color{incolor}In [{\color{incolor}2}]:} \PY{k+kn}{import} \PY{n+nn}{numpy}\PY{o}{,}\PY{n+nn}{utils}
        \PY{k+kn}{import} \PY{n+nn}{itertools}
        
        \PY{k}{def} \PY{n+nf}{getlogZ}\PY{p}{(}\PY{n}{w}\PY{p}{)}\PY{p}{:}
            \PY{n}{Z} \PY{o}{=} \PY{l+m+mi}{0}
            \PY{k}{for} \PY{n}{x} \PY{o+ow}{in} \PY{n}{itertools}\PY{o}{.}\PY{n}{product}\PY{p}{(}\PY{p}{[}\PY{o}{\PYZhy{}}\PY{l+m+mi}{1}\PY{p}{,} \PY{l+m+mi}{1}\PY{p}{]}\PY{p}{,} \PY{n}{repeat}\PY{o}{=}\PY{l+m+mi}{10}\PY{p}{)}\PY{p}{:}
                \PY{n}{Z} \PY{o}{+}\PY{o}{=} \PY{n}{numpy}\PY{o}{.}\PY{n}{exp}\PY{p}{(}\PY{n}{numpy}\PY{o}{.}\PY{n}{dot}\PY{p}{(}\PY{n}{x}\PY{p}{,}\PY{n}{w}\PY{p}{)}\PY{p}{)}
            \PY{k}{return} \PY{n}{numpy}\PY{o}{.}\PY{n}{log}\PY{p}{(}\PY{n}{Z}\PY{p}{)}
        
        \PY{n+nb}{print}\PY{p}{(}\PY{l+s+s1}{\PYZsq{}}\PY{l+s+si}{\PYZpc{}11.4f}\PY{l+s+s1}{\PYZsq{}}\PY{o}{\PYZpc{}}\PY{k}{getlogZ}(utils.w\PYZus{}small))
        \PY{n+nb}{print}\PY{p}{(}\PY{l+s+s1}{\PYZsq{}}\PY{l+s+si}{\PYZpc{}11.4f}\PY{l+s+s1}{\PYZsq{}}\PY{o}{\PYZpc{}}\PY{k}{getlogZ}(utils.w\PYZus{}medium))
        \PY{n+nb}{print}\PY{p}{(}\PY{l+s+s1}{\PYZsq{}}\PY{l+s+si}{\PYZpc{}11.4f}\PY{l+s+s1}{\PYZsq{}}\PY{o}{\PYZpc{}}\PY{k}{getlogZ}(utils.w\PYZus{}big))
\end{Verbatim}


    \begin{Verbatim}[commandchars=\\\{\}]
    18.2457
    89.5932
        inf

    \end{Verbatim}

    \begin{Verbatim}[commandchars=\\\{\}]
/usr/local/lib/python3.6/dist-packages/ipykernel\_launcher.py:7: RuntimeWarning: overflow encountered in exp
  import sys

    \end{Verbatim}

    We can observe from these results, that for parameter vectors with large
values (e.g. \texttt{utils.w\_big}), the exponential function overflows,
and thus, we do not obtain a correct value for the logarithm of
\texttt{Z}.

\begin{itemize}
\tightlist
\item
  Implement an improved function that avoids the overflow problem, and
  evaluate the partition function for the same parameters.
\end{itemize}

    We use the fact that: \[log\sum exp(x) = A + log\sum exp(x - A)\]

since:
\[A + log\sum exp(x - A) = A + log\sum exp(x) \cdot exp(- A) = A + log\cdot exp(- A)\sum exp(x) = log\sum exp(x)\]

    \begin{Verbatim}[commandchars=\\\{\}]
{\color{incolor}In [{\color{incolor}3}]:} \PY{k}{def} \PY{n+nf}{getlogZstable}\PY{p}{(}\PY{n}{w}\PY{p}{)}\PY{p}{:}
            \PY{n}{space\PYZus{}array} \PY{o}{=} \PY{n}{numpy}\PY{o}{.}\PY{n}{array}\PY{p}{(}\PY{n+nb}{list}\PY{p}{(}\PY{n}{itertools}\PY{o}{.}\PY{n}{product}\PY{p}{(}\PY{p}{[}\PY{o}{\PYZhy{}}\PY{l+m+mi}{1}\PY{p}{,} \PY{l+m+mi}{1}\PY{p}{]}\PY{p}{,} \PY{n}{repeat}\PY{o}{=}\PY{l+m+mi}{10}\PY{p}{)}\PY{p}{)}\PY{p}{)}
        
            \PY{n}{dot\PYZus{}prod} \PY{o}{=} \PY{n}{numpy}\PY{o}{.}\PY{n}{dot}\PY{p}{(}\PY{n}{w}\PY{p}{,} \PY{n}{space\PYZus{}array}\PY{o}{.}\PY{n}{T}\PY{p}{)}
            \PY{n}{A} \PY{o}{=} \PY{n}{dot\PYZus{}prod}\PY{p}{[}\PY{n}{numpy}\PY{o}{.}\PY{n}{argmax}\PY{p}{(}\PY{n}{dot\PYZus{}prod}\PY{p}{)}\PY{p}{]}
            
            \PY{n}{z} \PY{o}{=} \PY{n}{numpy}\PY{o}{.}\PY{n}{exp}\PY{p}{(}\PY{n}{dot\PYZus{}prod} \PY{o}{\PYZhy{}} \PY{n}{A}\PY{p}{)}
            \PY{k}{return} \PY{n}{A} \PY{o}{+} \PY{n}{numpy}\PY{o}{.}\PY{n}{log}\PY{p}{(}\PY{n}{z}\PY{o}{.}\PY{n}{sum}\PY{p}{(}\PY{p}{)}\PY{p}{)}
        
            
        \PY{n+nb}{print}\PY{p}{(}\PY{n}{getlogZstable}\PY{p}{(}\PY{n}{utils}\PY{o}{.}\PY{n}{w\PYZus{}small}\PY{p}{)}\PY{p}{)}
        \PY{n+nb}{print}\PY{p}{(}\PY{n}{getlogZstable}\PY{p}{(}\PY{n}{utils}\PY{o}{.}\PY{n}{w\PYZus{}medium}\PY{p}{)}\PY{p}{)}
        \PY{n+nb}{print}\PY{p}{(}\PY{n}{getlogZstable}\PY{p}{(}\PY{n}{utils}\PY{o}{.}\PY{n}{w\PYZus{}big}\PY{p}{)}\PY{p}{)}
\end{Verbatim}


    \begin{Verbatim}[commandchars=\\\{\}]
18.2457161254
89.5931912129
24921.9912806

    \end{Verbatim}

    \begin{itemize}
\tightlist
\item
  For the model with parameter \texttt{utils.w\_big}, evaluate the
  log-probability of the binary vectors contained in the list
  \texttt{itertools.product({[}-1,\ 1{]},\ repeat=10)}, and return the
  indices (starting from 0) of those that have probability greater or
  equal to 0.001.
\end{itemize}

    \begin{Verbatim}[commandchars=\\\{\}]
{\color{incolor}In [{\color{incolor}4}]:} \PY{n}{space\PYZus{}array} \PY{o}{=} \PY{n}{numpy}\PY{o}{.}\PY{n}{array}\PY{p}{(}\PY{n+nb}{list}\PY{p}{(}\PY{n}{itertools}\PY{o}{.}\PY{n}{product}\PY{p}{(}\PY{p}{[}\PY{o}{\PYZhy{}}\PY{l+m+mi}{1}\PY{p}{,} \PY{l+m+mi}{1}\PY{p}{]}\PY{p}{,} \PY{n}{repeat}\PY{o}{=}\PY{l+m+mi}{10}\PY{p}{)}\PY{p}{)}\PY{p}{)}
        \PY{n}{log\PYZus{}probs} \PY{o}{=} \PY{n}{numpy}\PY{o}{.}\PY{n}{dot}\PY{p}{(}\PY{n}{space\PYZus{}array}\PY{p}{,} \PY{n}{utils}\PY{o}{.}\PY{n}{w\PYZus{}big}\PY{p}{)} \PY{o}{\PYZhy{}} \PY{n}{getlogZstable}\PY{p}{(}\PY{n}{utils}\PY{o}{.}\PY{n}{w\PYZus{}big}\PY{p}{)}
        \PY{n}{probs} \PY{o}{=} \PY{n}{numpy}\PY{o}{.}\PY{n}{exp}\PY{p}{(}\PY{n}{log\PYZus{}probs}\PY{p}{)}
        
        \PY{k}{assert} \PY{o+ow}{not} \PY{n+nb}{any}\PY{p}{(}\PY{n}{probs}\PY{o}{\PYZgt{}}\PY{o}{=}\PY{l+m+mi}{1}\PY{p}{)} \PY{o+ow}{and} \PY{o+ow}{not} \PY{n+nb}{any}\PY{p}{(}\PY{n}{probs}\PY{o}{\PYZlt{}}\PY{l+m+mi}{0}\PY{p}{)}
        \PY{n}{numpy}\PY{o}{.}\PY{n}{where}\PY{p}{(}\PY{n}{probs} \PY{o}{\PYZgt{}}\PY{o}{=} \PY{l+m+mf}{0.001}\PY{p}{)}
\end{Verbatim}


\begin{Verbatim}[commandchars=\\\{\}]
{\color{outcolor}Out[{\color{outcolor}4}]:} (array([ 81,  83,  85,  87, 209, 211, 213, 215, 337, 339, 341, 343, 465,
                467, 469, 471, 597, 599, 725, 727, 853, 855, 981, 983]),)
\end{Verbatim}
            
    \subsection{Probability of generating data from a Gaussian model (30
P)}\label{probability-of-generating-data-from-a-gaussian-model-30-p}

Consider a multivariate Gaussian distribution of mean vector \texttt{m}
and covariance \texttt{S}. The probability associated to a vector
\texttt{x} is given by:

\[
p(\boldsymbol{x};(\boldsymbol{m},S)) = \frac{1}{\sqrt{(2\pi)^d \mathrm{det}(S)}} \exp \Big( - \frac12 (\boldsymbol{x}-\boldsymbol{m})^\top S^{-1} (\boldsymbol{x}-\boldsymbol{m})\Big)
\]

We consider the calculation of the probability of observing a certain
dataset

\[
\mathcal{D} = (\boldsymbol{x}^{(1)},\dots,\boldsymbol{x}^{(N)})
\]

assuming the data is generated according to a Gaussian distribution of
fixed parameters \(\boldsymbol{m}\) and \(S\). Such probability density
is given by the formula:

\[
\log P(\mathcal{D};(\boldsymbol{m},S)) = \log \prod_{i=1}^N p(\boldsymbol{x}^{(i)};(\boldsymbol{m},S))
\]

The function below implements such function:

    \begin{Verbatim}[commandchars=\\\{\}]
{\color{incolor}In [{\color{incolor}9}]:} \PY{k+kn}{import} \PY{n+nn}{numpy}\PY{o}{,}\PY{n+nn}{numpy}\PY{n+nn}{.}\PY{n+nn}{linalg}\PY{o}{,}\PY{n+nn}{utils}
        
        \PY{k}{def} \PY{n+nf}{logp}\PY{p}{(}\PY{n}{X}\PY{p}{,}\PY{n}{m}\PY{p}{,}\PY{n}{S}\PY{p}{)}\PY{p}{:}
            \PY{c+c1}{\PYZsh{} Find the number of dimensions from the data vector}
            \PY{n}{d} \PY{o}{=} \PY{n}{X}\PY{o}{.}\PY{n}{shape}\PY{p}{[}\PY{l+m+mi}{1}\PY{p}{]}
            
            \PY{c+c1}{\PYZsh{} Invert the covariance matrix}
            \PY{n}{Sinv} \PY{o}{=} \PY{n}{numpy}\PY{o}{.}\PY{n}{linalg}\PY{o}{.}\PY{n}{inv}\PY{p}{(}\PY{n}{S}\PY{p}{)}
            
            \PY{c+c1}{\PYZsh{} Compute the quadratic terms for all data points}
            \PY{n}{Q} \PY{o}{=} \PY{o}{\PYZhy{}}\PY{l+m+mf}{0.5}\PY{o}{*}\PY{p}{(}\PY{n}{numpy}\PY{o}{.}\PY{n}{dot}\PY{p}{(}\PY{n}{X}\PY{o}{\PYZhy{}}\PY{n}{m}\PY{p}{,}\PY{n}{Sinv}\PY{p}{)}\PY{o}{*}\PY{p}{(}\PY{n}{X}\PY{o}{\PYZhy{}}\PY{n}{m}\PY{p}{)}\PY{p}{)}\PY{o}{.}\PY{n}{sum}\PY{p}{(}\PY{n}{axis}\PY{o}{=}\PY{l+m+mi}{1}\PY{p}{)}
        
            \PY{c+c1}{\PYZsh{} Raise them quadratic terms to the exponential}
            \PY{n}{Q} \PY{o}{=} \PY{n}{numpy}\PY{o}{.}\PY{n}{exp}\PY{p}{(}\PY{n}{Q}\PY{p}{)}
            
            \PY{c+c1}{\PYZsh{} Divide by the terms in the denominator}
            \PY{n}{P} \PY{o}{=} \PY{n}{Q} \PY{o}{/} \PY{n}{numpy}\PY{o}{.}\PY{n}{sqrt}\PY{p}{(}\PY{p}{(}\PY{l+m+mi}{2}\PY{o}{*}\PY{n}{numpy}\PY{o}{.}\PY{n}{pi}\PY{p}{)}\PY{o}{*}\PY{o}{*}\PY{n}{d} \PY{o}{*} \PY{n}{numpy}\PY{o}{.}\PY{n}{linalg}\PY{o}{.}\PY{n}{det}\PY{p}{(}\PY{n}{S}\PY{p}{)}\PY{p}{)}
            
            \PY{c+c1}{\PYZsh{} Take the product of the probability of each data points}
            \PY{n}{Pprod} \PY{o}{=} \PY{n}{numpy}\PY{o}{.}\PY{n}{prod}\PY{p}{(}\PY{n}{P}\PY{p}{)}
            
            \PY{c+c1}{\PYZsh{} Return the log\PYZhy{}probability}
            \PY{k}{return} \PY{n}{numpy}\PY{o}{.}\PY{n}{log}\PY{p}{(}\PY{n}{Pprod}\PY{p}{)}
\end{Verbatim}


    Evaluation of this function for various datasets and parameters provided
in the file \texttt{utils.py} gives the following probabilities:

    \begin{Verbatim}[commandchars=\\\{\}]
{\color{incolor}In [{\color{incolor}10}]:} \PY{n+nb}{print}\PY{p}{(}\PY{n}{logp}\PY{p}{(}\PY{n}{utils}\PY{o}{.}\PY{n}{X1}\PY{p}{,}\PY{n}{utils}\PY{o}{.}\PY{n}{m1}\PY{p}{,}\PY{n}{utils}\PY{o}{.}\PY{n}{S1}\PY{p}{)}\PY{p}{)}
         \PY{n+nb}{print}\PY{p}{(}\PY{n}{logp}\PY{p}{(}\PY{n}{utils}\PY{o}{.}\PY{n}{X2}\PY{p}{,}\PY{n}{utils}\PY{o}{.}\PY{n}{m2}\PY{p}{,}\PY{n}{utils}\PY{o}{.}\PY{n}{S2}\PY{p}{)}\PY{p}{)}
         \PY{n+nb}{print}\PY{p}{(}\PY{n}{logp}\PY{p}{(}\PY{n}{utils}\PY{o}{.}\PY{n}{X3}\PY{p}{,}\PY{n}{utils}\PY{o}{.}\PY{n}{m3}\PY{p}{,}\PY{n}{utils}\PY{o}{.}\PY{n}{S3}\PY{p}{)}\PY{p}{)}
\end{Verbatim}


    \begin{Verbatim}[commandchars=\\\{\}]
-13.0067700574
-inf
-inf

    \end{Verbatim}

    \begin{Verbatim}[commandchars=\\\{\}]
/usr/local/lib/python3.6/dist-packages/ipykernel\_launcher.py:23: RuntimeWarning: divide by zero encountered in log

    \end{Verbatim}

    This function is numerically instable for multiple reasons. The product
of many probabilities, the inversion of a large covariance matrix, and
the computation of its determinant, are all potential causes for
overflow. Thus, we would like to find a numerically robust way of
performing each of these.

\begin{itemize}
\tightlist
\item
  Implement a numerically stable version of the function \texttt{logp}
\item
  Evaluate it on the same datasets and parameters as the function
  \texttt{logp}
\end{itemize}

    \begin{Verbatim}[commandchars=\\\{\}]
{\color{incolor}In [{\color{incolor}12}]:} \PY{k}{def} \PY{n+nf}{logp\PYZus{}stable}\PY{p}{(}\PY{n}{X}\PY{p}{,} \PY{n}{m}\PY{p}{,} \PY{n}{S}\PY{p}{)}\PY{p}{:}
             \PY{n}{N}\PY{p}{,} \PY{n}{d} \PY{o}{=} \PY{n}{X}\PY{o}{.}\PY{n}{shape}
             
             \PY{c+c1}{\PYZsh{} Take Penrose\PYZhy{}Moore\PYZhy{}Pseudo\PYZhy{}Inverse}
             \PY{n}{Sinv} \PY{o}{=} \PY{n}{numpy}\PY{o}{.}\PY{n}{linalg}\PY{o}{.}\PY{n}{pinv}\PY{p}{(}\PY{n}{S}\PY{p}{)}
             \PY{n}{Q} \PY{o}{=} \PY{o}{\PYZhy{}}\PY{l+m+mf}{0.5} \PY{o}{*} \PY{p}{(}\PY{n}{numpy}\PY{o}{.}\PY{n}{dot}\PY{p}{(}\PY{n}{X}\PY{o}{\PYZhy{}}\PY{n}{m}\PY{p}{,}\PY{n}{Sinv}\PY{p}{)}\PY{o}{*}\PY{p}{(}\PY{n}{X}\PY{o}{\PYZhy{}}\PY{n}{m}\PY{p}{)}\PY{p}{)}\PY{o}{.}\PY{n}{sum}\PY{p}{(}\PY{n}{axis}\PY{o}{=}\PY{l+m+mi}{1}\PY{p}{)}
             
             \PY{n}{sign}\PY{p}{,} \PY{n}{logdet} \PY{o}{=} \PY{n}{numpy}\PY{o}{.}\PY{n}{linalg}\PY{o}{.}\PY{n}{slogdet}\PY{p}{(}\PY{n}{S}\PY{p}{)}
             
             \PY{n}{Q} \PY{o}{=} \PY{p}{(}\PY{n}{numpy}\PY{o}{.}\PY{n}{dot}\PY{p}{(}\PY{n}{X}\PY{o}{\PYZhy{}}\PY{n}{m}\PY{p}{,}\PY{n}{Sinv}\PY{p}{)}\PY{o}{*}\PY{p}{(}\PY{n}{X}\PY{o}{\PYZhy{}}\PY{n}{m}\PY{p}{)}\PY{p}{)}\PY{o}{.}\PY{n}{sum}\PY{p}{(}\PY{n}{axis}\PY{o}{=}\PY{l+m+mi}{1}\PY{p}{)}
         
             \PY{k}{return} \PY{o}{\PYZhy{}} \PY{l+m+mf}{0.5} \PY{o}{*} \PY{p}{(}\PY{n}{N}\PY{o}{*}\PY{n}{d}\PY{o}{*}\PY{n}{numpy}\PY{o}{.}\PY{n}{log}\PY{p}{(}\PY{l+m+mi}{2}\PY{o}{*}\PY{n}{numpy}\PY{o}{.}\PY{n}{pi}\PY{p}{)} \PY{o}{+} \PY{n}{N}\PY{o}{*}\PY{n}{sign}\PY{o}{*}\PY{n}{logdet} \PY{o}{+} \PY{n}{Q}\PY{o}{.}\PY{n}{sum}\PY{p}{(}\PY{p}{)}\PY{p}{)}
         
         \PY{n+nb}{print}\PY{p}{(}\PY{n}{logp\PYZus{}stable}\PY{p}{(}\PY{n}{utils}\PY{o}{.}\PY{n}{X1}\PY{p}{,}\PY{n}{utils}\PY{o}{.}\PY{n}{m1}\PY{p}{,}\PY{n}{utils}\PY{o}{.}\PY{n}{S1}\PY{p}{)}\PY{p}{)}
         \PY{n+nb}{print}\PY{p}{(}\PY{n}{logp\PYZus{}stable}\PY{p}{(}\PY{n}{utils}\PY{o}{.}\PY{n}{X2}\PY{p}{,}\PY{n}{utils}\PY{o}{.}\PY{n}{m2}\PY{p}{,}\PY{n}{utils}\PY{o}{.}\PY{n}{S2}\PY{p}{)}\PY{p}{)}
         \PY{n+nb}{print}\PY{p}{(}\PY{n}{logp\PYZus{}stable}\PY{p}{(}\PY{n}{utils}\PY{o}{.}\PY{n}{X3}\PY{p}{,}\PY{n}{utils}\PY{o}{.}\PY{n}{m3}\PY{p}{,}\PY{n}{utils}\PY{o}{.}\PY{n}{S3}\PY{p}{)}\PY{p}{)}
\end{Verbatim}


    \begin{Verbatim}[commandchars=\\\{\}]
-13.0067700574
-1947.97098067
-218646.173856

    \end{Verbatim}


    % Add a bibliography block to the postdoc
    
    
    
    \end{document}
